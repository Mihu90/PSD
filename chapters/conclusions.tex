\chapter{Concluzii}

În cazul unui forum datele se protejează nu sunt foarte multe și atât de importante pentru a aloca foarte multe resurse și eventual pentru a diminua performanța întregului sistem doar pentru a asigura o securitate foarte restrictivă și agresivă cum ar fi în cazul unui cont de administrare a cardului bancar.

Unele măsuri de securitate nu este eficient a se implementa în cazul unui forum deoarece pot degrada semnificativ performanța sau pot introduce multe alte breșe de securitate dacă nu se acordă o atenție sporită la impelementarea lor sau la aplicarea lor de către administratori. În cazul mai multor variante de securitate s-a ales să se facă un compromis din punct de vedere al securității pentru a putea oferi sistemului o performanță mai bună.

Este de ajuns de multe ca metodă simplă de protecție să fie ascunsă ascunderea versiunii platformei, astfel unui atacator i-ar fi mult mai greu să găsească platformele care utilizează o versiune care are un anume punct slab.

Ca și modalități de autentificare și verificare a identității sunt de ajuns metodele expuse în capitolele anterioare. Astfel nu este nevoie de coduri trimise prin sms pentru a fi introduse la autentificare cum este în cazul autentificării într-un sistem de home banking, datele fiind de un nivel mai puțin important în cazul unui forum. 

Blocarea autentificării dupa mai multe încercări eșuate în cazul unui utilizator este un punct de ajuns pentru prevenirea atacurilor de tip brute force. Astfel se îngreunează mult procesul de reușită al unui atacator de acest gen. Dacă se ia în considerare și posibila îmbunătățire cu referire la locul de păstrare al datelor despre câte tentative de autentificare au avut loc nemaiputându-se șterge cookieurile pentru by passarea acestei breșe atunci se asigură o securitate destul de ridicată sistemului.
\newgeometry{margin=5cm}
\thispagestyle{empty}
\begin{multicols}{2}
[
\section*{Cuvânt înainte,}
]
\footnotesize {
În ziua de astăzi, internetul este bazat pe o structură avansată hardware și software ce se află într-o continuă dezvoltarea. Domeniul IT, în general, nu a fost la saturație și nici nu pare să ajungă în următorii zeci de ani. Din ce în ce mai multe dispozitive sunt interconectate, din ce în ce mai multe lucruri sunt realizate cu ajutorul lui. Pe zi ce trece, informația care este pusă la dispoziție crește din punct de vedere al cantității, fiind disponibilă utilizatorului prin intermediul site-urilor web. \textbf{Forumul} reprezintă un mediu virtual, prin care utilizatorii pot discuta diverse lucruri, acoperind o plajă foarte mare de subiecte din diverse domenii.

Din păcate, o dată cu trecerea timpului, forumurile au început să nu mai fie la modă. Au avut un apogeu undeva pe la începutul anilor 2010.
Tendința din ultimii doi ani, a fost o scădere drastică a numărului de noi forumuri înființate, probabil și pe fondul apariției a tot mai multor bloguri și rețele sociale. Nu avem de-a face decât cu o consecință a foaptului că servicile de găzduire web au devenit din ce în ce mai accesibile, iar cunoștiințele necesare administrării unui forum regăsindu-se foarte ușor prin intermediul unor articole / tutoriale.

Asta nu înseamnă că forumurile au dispărut. Cele deja existente și care activează pe o anumită nișă, sunt încă la mare căutare. Ele concentrează o cantitate destul de mare de informație sensibilă, într-adevăr nu la fel de sensibilă ca cea expusă de o bancă sau un magazin virtual, dar care poate fi folosită în mod indirect pentru obținerea altor informații. De ceea partea de securitate reprezintă încă o provocare.
}
\end{multicols}

\vspace{1cm}

\begin{center}
    \qrcode[height=3cm]{http://www.mybb.com}
\end{center}

\restoregeometry
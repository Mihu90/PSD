\chapter{Introducere}

În domeniul IT, forumul reprezintă un site online, bazat pe o platformă specifică prin care aduce laolaltă o populație mare de internauți, la fel ca la o adunare. Problemele care se discută sunt dintr-o serie de domenii. Mai mult decât atât, există la ora actuală o serie de servicii web care sunt oferite prin extensia unui platforme de forumuri. Putem să luăm un simplu exemplu și anume cel al unei firme mici ce dezvoltă modificări sau teme pentru alte platforme web. În acest caz, forumul este o unealtă importantă în oferirea unui support corespunzător atât celor care au cumpărat produsul respectiv cât și pentru viitorii cumpărători.

Într-adevăr, o astfel de platformă nu va stoca informații foarte importante cu privire la utilizatorii săi - precum informații despre cardurile bancare, dar o breșă în sistem poate produce pagube destul de însemnate. Ca să înțelegem mai bine putem compara compromiterea unui forum cu cea a unui blog. Compromiterea unui blog poate duce la scurgerea de informații cu privire la câteva zeci sau sute de utilizatori, pe când la un forum vorbim la o scară mult mai mare, de mii sau milioane de conturi. Acest lucru, colaborat și cu ideea că majoritatea utilizatorilor au acxeleași credențiale pe toate platformele pe care activează, conduce la daune din ce în ce mai importante o dată cu coruperea datelor dintr-un loc.

La ora actuală, pe piață există o serie de jucători ce dezvoltă platforme de forumuri. Există atât platforme open source cât și o serie de platforme comerciale. Una dintre cele mai importante platforme open source o reprezintă MyBB. De ce această platformă? Pentru că are o vechime considerabilă, prima versiunea apărând în anul 2002. Totodată, a fost desemnată cea mai bună platformă open-source de către \textit{forum-software.org} în anii 2008, 2010. 2011, 2012 și 2013.

Acest proiect își propune realizarea unei scurte analize în ceea ce privește politica de securitate a acestei platforme. Am dori să precizăm că nicio platformă nu are o politică de securitate explicită, dar ea poate fi dedusă în funcție de o serie de factori.

Ca și structură generală a proiectului, ne propunem mai întâi să vorbim despre câteva aspecte pe care platforma le evidențiază, aspecte ce definesc politica adoptată; după care am dori să prezentăm câteva aspecte ce ar mai putea fi îmbunătățite, aspecte atât subiective cât și obiective. În final, va trebui să existe și o retrospectivă, un sumar al ideilor anunțate în capitolele anterioare.